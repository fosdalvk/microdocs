%!TEX TS-program = XeLaTeX
%!TEX encoding = UTF-8 Unicode
\documentclass[namedreferences]{autons}

\newcommand{\theId}{0x00C}
\newcommand{\theVersion}{0.1}
\newcommand{\theTitle}{The Karush-Kuhn-Tucker Conditions of Optimality}
\newcommand{\theKeywords}{kkt, optimization, conditions of optimality, lagrange, lagrangian}
\newcommand{\theAbstract}{In mathematics, the Karush–Kuhn–Tucker conditions (also known as the Kuhn-Tucker or KKT conditions) are necessary for a solution in nonlinear programming to be optimal, provided some regularity conditions are satisfied. It is a generalization of the method of Lagrange multipliers to inequality constraints. The conditions are named after William Karush, Harold W. Kuhn, and Albert W. Tucker.}

%%%%%%%%%%%%%%%%%%%%%%%%%%%%%%%%%%%%%%%%%%%%%%%%%%%%%%%%%%%%%%%%%%%%%%%%%%%%%%%%
%. The Preamble...
\def\copyrightline{\copyright\- Nima Talebi 2010}
\newcommand{\theFirstName}{Nima}
\newcommand{\theLastName}{Talebi}
\newcommand{\theEmail}{nima@autonomy.net.au}
\newcommand{\theSubtitle}{Microdocs - Article {\tt\theId}}
\newcommand{\theInstitute}{Autonomy Corporation Pty Ltd}
\newcommand\theAuthor{\theFirstName \theLastName}
%-------------------------------------------------------------------------------
\usepackage[usenames,dvipsnames]{color}
\usepackage[
    pdftitle={\theTitle},
    pdfsubject={\theSubtitle},
    pdfauthor={\theAuthor},
    pdfcreator={\theAuthor},
    pdfkeywords={\theKeywords},
    backref,
    hyperfigures=true,
    final,
    bookmarks=false,
    xetex,
    dvipdfm,
    breaklinks,
    bookmarksopen=true,
    bookmarksnumbered=true,
    colorlinks=true,
    linkcolor=Cyan,
    anchorcolor=Red,
    citecolor=ForestGreen,
    filecolor=Aquamarine,
    menucolor=Orange,
    runcolor=Red,
    urlcolor=LimeGreen
]{hyperref}
\InputIfFileExists{preamble.tex}{}{}

%%%%%%%%%%%%%%%%%%%%%%%%%%%%%%%%%%%%%%%%%%%%%%%%%%%%%%%%%%%%%%%%%%%%%%%%%%%%%%%%
%. The Document...
\begin{document}
\begin{article}
\begin{opening}
    \title{\theTitle}
    \runningtitle{\theTitle}
    \subtitle{\theSubtitle}

    \runningauthor{\theAuthor}
    \author{\theFirstName\ \surname{\theLastName}\email{\theEmail}}
    \institute{\theInstitute}
    \InputIfFileExists{authors.tex}{}{}

    \IfFileExists{motto.tex}{
        \begin{motto}[prose]
            The only thing that never looks right is a rule. There is not in existence a page with a rule on it that cannot be instantly and obviously improved by taking the rule out.  \rightline{George Bernard Shaw, In {\it The Dolphin} (1940)}


        \end{motto}
    }{}

    \begin{abstract}
        \theAbstract
    \end{abstract}
    \keywords{\theKeywords}

    \InputIfFileExists{abbrev.tex}{}{}
    \InputIfFileExists{nomen.tex}{}{}
    \InputIfFileExists{dedication.tex}{}{}
    \InputIfFileExists{classify.tex}{}{}
    \classification{Microdoc}{\theId}
    \classification{Version}{\theVersion}
\end{opening}

%-------------------------------------------------------------------------------
%. Finish off...
\addtocounter{tocdepth}{1}
\tableofcontents

%-------------------------------------------------------------------------------

\section{Notation}

%-------------------------------------------------------------------------------
%-------------------------------------------------------------------------------
\pagebreak
\IfFileExists{local.bib}{
    \bibliography{autonomy,local}
}{
    \bibliography{autonomy}
}

%%%%%%%%%%%%%%%%%%%%%%%%%%%%%%%%%%%%%%%%%%%%%%%%%%%%%%%%%%%%%%%%%%%%%%%%%%%%%%%%
\end{article}
\bibliographystyle{alpha}
\end{document}
\endofdocument

