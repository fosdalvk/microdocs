%!TEX TS-program = XeLaTeX
%!TEX encoding = UTF-8 Unicode
\documentclass[namedreferences]{autons}

\newcommand{\theId}{0x009}
\newcommand{\theVersion}{0.1}
\newcommand{\theTitle}{Classes of Mathematical Results}
\newcommand{\theKeywords}{theorem, corollary, proposition, lemma, axiom, perspective, results}
\newcommand{\theAbstract}{Consise definitions where possible, and otherwise usage guidelines of the various classes of mathematical results found in mathematics literature.}

%%%%%%%%%%%%%%%%%%%%%%%%%%%%%%%%%%%%%%%%%%%%%%%%%%%%%%%%%%%%%%%%%%%%%%%%%%%%%%%%
%. The Preamble...
\def\copyrightline{\copyright\- Nima Talebi 2010}
\newcommand{\theFirstName}{Nima}
\newcommand{\theLastName}{Talebi}
\newcommand{\theEmail}{nima@autonomy.net.au}
\newcommand{\theSubtitle}{Microdocs - Article {\tt\theId}}
\newcommand{\theInstitute}{Autonomy Corporation Pty Ltd}
\newcommand\theAuthor{\theFirstName \theLastName}
%-------------------------------------------------------------------------------
\usepackage[usenames,dvipsnames]{color}
\usepackage[
    pdftitle={\theTitle},
    pdfsubject={\theSubtitle},
    pdfauthor={\theAuthor},
    pdfcreator={\theAuthor},
    pdfkeywords={\theKeywords},
    backref,
    hyperfigures=true,
    final,
    bookmarks=false,
    xetex,
    dvipdfm,
    breaklinks,
    bookmarksopen=true,
    bookmarksnumbered=true,
    colorlinks=true,
    linkcolor=Cyan,
    anchorcolor=Red,
    citecolor=ForestGreen,
    filecolor=Aquamarine,
    menucolor=Orange,
    runcolor=Red,
    urlcolor=LimeGreen
]{hyperref}
\InputIfFileExists{preamble.tex}{}{}

%%%%%%%%%%%%%%%%%%%%%%%%%%%%%%%%%%%%%%%%%%%%%%%%%%%%%%%%%%%%%%%%%%%%%%%%%%%%%%%%
%. The Document...
\begin{document}
\begin{article}
\begin{opening}
    \title{\theTitle}
    \runningtitle{\theTitle}
    \subtitle{\theSubtitle}

    \runningauthor{\theAuthor}
    \author{\theFirstName\ \surname{\theLastName}\email{\theEmail}}
    \institute{\theInstitute}
    \InputIfFileExists{authors.tex}{}{}

    \IfFileExists{motto.tex}{
        \begin{motto}[prose]
            The only thing that never looks right is a rule. There is not in existence a page with a rule on it that cannot be instantly and obviously improved by taking the rule out.  \rightline{George Bernard Shaw, In {\it The Dolphin} (1940)}


        \end{motto}
    }{}

    \begin{abstract}
        \theAbstract
    \end{abstract}
    \keywords{\theKeywords}

    \InputIfFileExists{abbrev.tex}{}{}
    \InputIfFileExists{nomen.tex}{}{}
    \InputIfFileExists{dedication.tex}{}{}
    \InputIfFileExists{classify.tex}{}{}
    \classification{Microdoc}{\theId}
    \classification{Version}{\theVersion}
\end{opening}

%-------------------------------------------------------------------------------
%. Finish off...
\addtocounter{tocdepth}{1}
\tableofcontents

%-------------------------------------------------------------------------------

\section{Hypothesis}
A hypothesis/conjecture = an unproved statement usually stated when their proof is unknown, or when a possible proof is out of the scope of the work at hand

\section{Corollaries}
A corollary is a proposition that follows from (and is often appended to) one already proven; a direct or natural consequence or result.  For example, ``{\it the huge increases in unemployment were the corollary of expenditure cuts}''.

\section{Fact}
a "fact" is a small result or small statement that is known to be true already, but is stated to give more background of the following work. Sometimes a fact might be proven, and used in the same way as a lemma would be.

Corollaries are usually practiced in two situations:
\begin{enumerate}
    \item Another way to state the result.
    \item A consequence of the theorem.
\end{enumerate}

A {\it practical} consequence that follows naturally: {\it blind jealousy is a frequent corollary of passionate love}.

Corollaries typically aren't proved, because the proof is self-evident\footnote{A proof should be provided for clarity where required}.

\section{Propositions}
A proposition = an unproved statement that may be proved later in the paper, or gives a sort of goal of the following content
someone usually calls a proposition a conjecture, when they really really believe it's true, but they can't prove it

\section{Conjencture}
A conjecture is a proposition that is unproven but appears correct and has not been disproven. Karl Popper pioneered the use of the term "conjecture" in scientific philosophy. Conjecture is contrasted by hypothesis (hence theory, axiom, principle), which is a testable statement based on accepted grounds. In mathematics, a conjecture is an unproven proposition or theorem which appears correct.

\section{Theorems}
Theorems are named as such to signify that they are useful in proving other theorems, or to signify they have a bountyful of useful corollaries.

\section{Lemma}
Lemmas should be stated before the theorem is stated, and they act as theorem building blocks.

Lemmas are essentially equivalent to theorems, however they are generally used in different ways. They are used to build up results, and the final result is promoted toa theorem.  Lemmas signify they are useful in proving usually a smaller selection or even a particular theorem.

\section{Proofs}
After theorems have been delivered, they, along with their component lemmas, are used in completing the final proof.

\section{Axioms}
 A rule that is true by definition. Axioms are the most basic thing, and they define the mathematical system at hand. All deductions come from axioms.  Axioms are assumed truths which we deduct theorems from.  They are like the building blocks for the mathematical system we will work in.  They don't require any proving, but a set of axioms must be consistent with each other, and a lemma is deduced from axioms.  Axioms are true by definition, they are not results.

\section{Perspective}
ok, how about this perspective then.  So a theorem is nothing more than a logical deduction from your axioms no?
so, isn't it the important part the axioms of the language?  and not the theorems?
correct, according to me


%-------------------------------------------------------------------------------
%-------------------------------------------------------------------------------
\pagebreak
\IfFileExists{local.bib}{
    \bibliography{autonomy,local}
}{
    \bibliography{autonomy}
}

%%%%%%%%%%%%%%%%%%%%%%%%%%%%%%%%%%%%%%%%%%%%%%%%%%%%%%%%%%%%%%%%%%%%%%%%%%%%%%%%
\end{article}
\bibliographystyle{alpha}
\end{document}
\endofdocument

